% !Mode:: "TeX:UTF-8"
%!TEX program  = xelatex

\documentclass[withoutpreface,bwprint]{cumcmthesis}

\title{基于汽车动力学响应分析的减速路障设计}
\tihao{C}
\baominghao{19}
\begin{document}
\maketitle
\begin{abstract}

减速路障在道路上寻常可见,也是不可缺少的交通设施。
本文基于车辆通过减速路障时的汽车动力学响应而设计几种可行的形状,并讨论车型和车速对减速效果和车辆颠簸程度的影响。

\keywords{减速丘\quad 汽车动力学\quad }
\end{abstract}

\section{问题重述}

为了道路交通安全,常在一些需要控制车速路段设置减速路障,
但减速路障会影响车内乘客的体验,造成车辆的颠簸,甚至损伤车辆。
因此设计既能有效降低车速、又最低程度影响驾驶员的驾驶体验的减速路障具有重要的现实意义。

减速路障的形状、间距以及车速、车型对减速效果和车辆颠簸程度有不同程度的影响。
试对问题分析后,设计出合适的形状和间距,并讨论车型和车速对减速效果和车辆颠簸程度的影响。

\section{问题分析}

减速路障是通过影响驾驶人的驾驶心理从而实现减速的。当车辆以较高的车速通过减速路障时,
剧烈的震动会从轮胎经由车身和座椅传递给驾驶人,产生垂直方向的加速度,产生强烈的不适感\citeup{cmhskl2008}。
通常,驾驶人会认为驾驶舒适度越小,道路情况越危险,车辆行驶安全性更小。
因此,减速路障的设置会促使驾驶人主动降低速度,选择较低的期望车速。
在期望车速指导下,驾驶员将主动驾驶车辆以较低的速度通过减速路障。

\subsection{问题一分析}

不同形状的减速路障的减速效果主要体现在能否引起驾驶员注意及驾驶员以较高速度通过时的减速效果,
不同形状的减速路障也会带给驾驶人及乘车人不同的行车体验。

分析不同形状的减速路障,把汽车近似成为一个刚体,分析

\subsection{问题二分析}

\subsection{问题三分析}

\section{模型假设}

\section{符号说明}

\begin{tabular}{cc}
 \hline
 \makebox[0.4\textwidth][c]{符号}	&  \makebox[0.5\textwidth][c]{意义} \\ \hline
 $D$	    & 木条宽度(cm) \\ \hline
 $L$	    & 木板长度(cm)  \\ \hline
\end{tabular}

\section{模型建立与求解}

\section{模型检验及结果分析}

\section{模型评价与改进方向}

%参考文献
\bibliographystyle{plain}
\nocite{*}
\bibliography{reference}

\newpage
%附录
\appendix
\section{Hello 源代码}
\lstinputlisting[language=C]{src/hello.c}


\end{document} 