% !Mode:: "TeX:UTF-8"
%!TEX program  = xelatex

\documentclass[withoutpreface,bwprint]{cumcmthesis}

%\title{基于汽车动力学响应分析的减速路障设计}
\title{基于推广 \LaTeX 的减速路障设计}
\tihao{C}
\baominghao{19}
\begin{document}
\maketitle
\begin{abstract}

减速路障在道路上寻常可见,也是不可缺少的交通设施。
本文基于车辆通过减速路障时的汽车动力学响应而设计几种可行的形状,并讨论车型和车速对减速效果和车辆颠簸程度的影响。

在解决问题前,先通过车速的数学期望给出了减速带的减速效果的定量描述,
并用通过减速带时车内乘客全身振动的总加权加速度的均方根值来评价人体舒适性。

\keywords{减速路障\quad 交通稳静化\quad 汽车动力学}
\end{abstract}

\section{问题重述}

在社区或学校校园道路上,机动车、非机动车、行人混行严重,
加之行人和驾驶员的安全意识较差,
致使支路和社区、学校周边存在严重的交通安全隐患。

为了道路交通安全,常在一些需要控制车速路段设置减速路障。
但减速路障会影响车内乘客的体验,造成车辆的颠簸,甚至损伤车辆。
因此设计既能有效降低车速、又最低程度影响驾驶员的驾驶体验的减速路障具有重要的现实意义。

减速路障的形状、间距以及车速、车型对减速效果和车辆颠簸程度有不同程度的影响。
试对问题建模分析后,设计出合适的形状和间距,并讨论车型和车速对减速效果和车辆颠簸程度的影响。

\section{问题分析}

\subsection{引言}

%\subsubsection{减速带分类及减速原理}

交通平静化可分为限制机动车行驶速度、限制机动车交通量和限制路侧停车三大类方法,
其中限制机动车行驶速度分为垂直式、水平式和路宽缩减式。

缩短路宽式路障是一种通过放置路障减小路宽,使司机因为心理因素而主动减速,达到减速的效果。
但该种路障会大大减小路面车流量,甚至会造成交通拥堵,仅适合车流量小的路段使用。
故本文只讨论垂直式减速路障。

减速带是垂直速度控制措施之一,是通过影响驾驶人的驾驶心理从而实现减速的。当车辆以较高的车速通过减速路障时,
剧烈的震动会从轮胎经由车身和座椅传递给驾驶人,产生垂直方向的加速度,产生强烈的不适感\citeup{cmhskl2008h}。
通常,驾驶人会认为驾驶舒适度越小,道路情况越危险,车辆行驶安全性更小。
因此,减速路障的设置会促使驾驶人主动降低速度,选择较低的期望车速。
在期望车速指导下,驾驶员将主动驾驶车辆以较低的速度通过减速路障\citeup{hanyan2009jsdjsylh}。

\subsection{问题一分析}

通常意义下,减速路障的形状应该包括其截面形状、俯视投影形状以及其在道路中的安装位置等。
限于篇幅,本文中的形状均指减速路障一个竖直截面的形状。

不同形状的减速路障的减速效果主要体现在能否引起驾驶员注意及驾驶员以较高速度通过时的减速效果,
不同形状的减速路障也会带给驾驶人及乘车人不同的行车体验。

分析不同形状的减速路障,把汽车近似看作一个刚体,分析其受力与加速度变化情况,定性、定量地分析
减速路障的形状对减速效果与舒适度的影响。

\subsection{问题二分析}

减速路障的间距也是决定路段减速效果的一个重要因素。如果间距设置过小,则驾驶舒适度很小;
反之,若间距设置过大,则驾驶员可以在两个减速路障之间加速,无法达到路段要求的减速效果。

分析汽车在通过若干个间距相同的路障时速度的周期性变化,来找出通过减速路障过程中
速度变化最平稳的情况,寻找最合适的路障摆放间距。

\subsection{问题三分析}

经过对路上车辆的观察,通过减速带时车轮的直径、离心率和车辆种类不同,减速效果和乘坐感受也会有明显的不同。
建立表格,横向分析不同车速以及不同车型压过减速路障时的速度变化及重心、加速度变化,总结规律,根据
路上车流量较大的车型来放置合适的减速路障。

\section{模型假设}

\begin{enumerate}

\item 假设汽车前轮和后轮分别压过减速路障时可以作为刚体。%,质量为车和人的总质量的二分之一(定值)。

\item 假设人在汽车内姿势不发生变化,车胎无减震效果。

\item 假设减速路障发生微小形变(可忽略)。

\end{enumerate}

\section{符号说明}

\begin{tabular}{cc}
 \hline
 \makebox[0.4\textwidth][c]{符号}	&  \makebox[0.5\textwidth][c]{意义} \\ \hline
 $\Xi$      & 减速带 \\ \hline
 $f_{\Xi}(v)$ & 车辆通过减速带 $\Xi$ 时车速的概率分布函数 \\ \hline
 $\zeta$    & 减速带减速效果   \\ \hline
 $\nu_{m}$  & 路段建议行使速度(m/s) \\ \hline
 $a_w(\tau)$      & 汽车通过减速带某时刻总加权加速度的均方根值 \\ \hline
 $\kappa$   & 汽车乘坐舒适度 \\ \hline
% $M$	    & 车和人的总质量(kg) \\ \hline
% $\Delta L$	    & 相邻两个减速路障间距(m)  \\ \hline
% $v_0$      & 车辆初始速度(km/h) \\ \hline
% $h$        & 路障的高度 (cm) \\ \hline
% $r$        & 轮胎半径 (cm)\\ \hline
% $\theta$   & 轮胎接触点切线与水平面夹角 (rad) \\ \hline
% $\alpha$   & 车体与地面夹角 (rad) \\ \hline
\end{tabular}

\section{模型建立与求解}

\subsection{相关指标的定量描述}

\subsubsection{减速效果}

车辆通过减速带的减速由两部分组成,
一是心理作用下驾驶员的主动减速,
二是当车辆压过减速带时滚动摩擦力产生的阻力矩迫使车辆减速。
在实际道路中,一般情况下都是由驾驶员主动减速到道路限速,物理减速的比例与主动减速相比微乎其微,在本文中忽略不计。

如果以某一速度驶过减速带时驾驶体验最佳,
那么大部分驾驶员会主动降低到这个速度,同时大部分驾驶员也希望尽可能地提高驾驶速度。
设车辆驶过减速带 $\Xi$ 时速度的概率密度为 $f_{\Xi}(v)$,则平均车速为
\begin{equation}
\bar{v} = \int_{0}^{+\infty}{vf_{\Xi}(v)\mathrm{d}v}
\end{equation}

如果这个速度与当前道路期望的车辆行驶速度越接近,那么就认为这个减速带的减速效果越好,
即定义减速效果为
\begin{equation}
\zeta = \left|\frac{\bar{v}-\nu_m}{\nu_m}\right|
\end{equation}
$\zeta$ 越小,则减速效果越好。

\subsubsection{驾驶体验}

%驾驶体验受多个因素影响。

车辆在通过减速带时,%车体会获得一个竖直方向上的加速度,人的重心突然上升,表现为“颠簸”。
%一般来说竖直加速度越大,则主观上颠簸越厉害。
会使车内乘客处于振动环境中。本文主要讨论全身振动对驾驶体验的影响。

国际标准化组织(ISO)于 1997 年正式发布了“ISO 2631-1: 1997 《人体承受全身振动的评价》”,
其中给出了人体舒适性评价的近似法,即用总加权加速度的均方根值 $a_w$ 来评价人体舒适性\citeup{iso263111997}
\[
    a_w = \sqrt{(1.4a_{xw})^2 + (1.4a_{yw})^2 + a_{zw}^2}
\]
在通过减速带时,一般可认为 $a_{yw} = 0$,即
\begin{equation}
a_w = \sqrt{(1.4a_{xw})^2 + a_{zw}^2}
\end{equation}

该评价还给出人主观感受有些不舒适的最大加权加速度约为 $a_{wm} \approx 1.0\,m\cdot{}s^{-2}$\citeup{iso263111997},
于是可以定义汽车乘坐舒适度为
\begin{equation}
\kappa = - \lg{\frac{\displaystyle\max_{0\leq\tau\leq{}t}{\{a_w(\tau)\}}}{a_{wm}}}
\end{equation}
其中 $t$ 为汽车通过减速带的总时间。

当 $\kappa > 0$ 时,则表明人乘坐汽车通过减速带时没有强烈的不适感。

\subsection{减速路障几何形状设计与对比}

\section{模型检验及结果分析}

\section{模型评价与改进方向}

\subsection{改进方向1}
% 质量 

\subsection{改进方向2}
% 减震装置 


%参考文献
\bibliographystyle{unsrt}
\nocite{*}
\bibliography{reference}

\newpage
%附录
\appendix
%\section{Hello 源代码}
%\lstinputlisting[language=C]{src/hello.c}


\end{document} 