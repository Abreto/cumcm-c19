% !Mode:: "TeX:UTF-8"
%!TEX program  = xelatex

\documentclass[withoutpreface,bwprint]{cumcmthesis}

\title{基于最佳舒适度和最好减速效果的减速路障设计}
\tihao{C}
\baominghao{19}
\begin{document}
\maketitle
\begin{abstract}

\keywords{}
\end{abstract}

\section{问题重述}

为了道路交通安全,常在一些需要控制车速路段设置减速路障,
但减速路障会影响车内乘客的体验,造成车辆的颠簸,甚至损伤车辆。
因此设计既能有效降低车速、又最低程度影响驾驶员的驾驶体验的减速路障具有重要的现实意义。

减速路障的形状、间距以及车速、车型对减速效果和车辆颠簸程度有不同程度的影响。
试对问题分析后,设计出合适的形状和间距,并讨论车型和车速对减速效果和车辆颠簸程度的影响。

\section{问题分析}

由于减速路障形状、间距不同,导致车辆轮胎压过路障时,发生的形变以及受力方向都会发生变化
\subsection{问题一分析}

\subsection{问题二分析}

\subsection{问题三分析}

\section{模型假设}

\section{符号说明}

\begin{tabular}{cc}
 \hline
 \makebox[0.4\textwidth][c]{符号}	&  \makebox[0.5\textwidth][c]{意义} \\ \hline
 D	    & 木条宽度(cm) \\ \hline
 L	    & 木板长度(cm)  \\ \hline
 W	    & 木板宽度(cm)  \\ \hline
 N	    & 第n根木条  \\ \hline
 T	    & 木条根数  \\ \hline
 H	    & 桌子高度(cm)  \\ \hline
 R	    & 桌子半径(cm)  \\ \hline
 R	    & 桌子直径(cm)  \\ \hline
\end{tabular}

\section{模型建立与求解}

\section{模型检验及结果分析}

\section{模型评价与改进方向}

%参考文献
\bibliographystyle{plain}
\nocite{*}
\bibliography{reference}

\newpage
%附录
\appendix
\section{Hello 源代码}
\lstinputlisting[language=C]{src/hello.c}


\end{document} 